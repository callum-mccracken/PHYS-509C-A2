\section{COVID-19 Study}
\begin{itemize}
    \item Sample size: 3330 people in Santa Clara County, California
    \item $N_{+,t}$ = 50 positive test results in test group
    \item Control group 1: 3324 people, definitely negative
    \item $N_{+,c1}$ = 16 positives in control 1
    \item Control group 2: 157 people, definitely negative
    \item $N_{+,c2}$ = 130 positives in control 2 (27 false negatives)
\end{itemize}

Calculate the Bayesian 95\% central interval on the fraction of people in Santa Clara County who actually had antibodies for COVID-19, marginalizing over the false positive and false negative rates. Assume flat priors on all parameters. Submit a plot of the posterior distribution for the true incidence rate as well as your code or calculation.

We want to find the distribution of the true positive rate in the population (and then get the 95\% interval), $P(\text{real pos})$.

We have positive test probabilities.

A positive test could have come from a true positive or a false positive
\begin{align*}
    P(\text{test, pos}) &= P(\text{test, true pos})P(\text{real, pos}) + P(\text{test, false pos})P(\text{real, neg}) \\
    &= P(\text{test, true pos})P(\text{real, pos}) + P(\text{test, false pos})(1 - P(\text{real, pos})) \\
\end{align*}

\begin{align*}
    \frac{P(\text{test, pos}) - P(\text{test, false pos})}{P(\text{test, true pos}) - P(\text{test, false pos})} &= P(\text{real, pos}) \\
\end{align*}
The question says to marginalize over false positive and false negative rates, write it in terms of those

\begin{align*}
    \frac{P(\text{test, pos}) - P(\text{test, false pos})}{1-P(\text{test, false neg}) - P(\text{test, false pos})} &= P(\text{real, pos}) \\
\end{align*}

And rewrite using shorter variable names (most variable names should be clear, I used $A$ for antibodies, $F$=False, $N$=negative, $P$=positive):
\begin{align*}
    \frac{P(P) - P(FP)}{1-P(FN) - P(FP)} &= P(A) \\
\end{align*}


The priors: 

$P(A) = P(P) = P(FP) = P(FN) = 1$ (uniform between 0 and 1)

Likelihood of seeing our data:

(Here our data is $P(FP) = \frac{16}{3324}, P(FN) = \frac{27}{157}$)
\begin{align*}
    P(D|R_{P}, R_{FP}, R_{FN}) = P(D_1|R_{P}, R_{FP}, R_{FN})P(D_2|R_{P}, R_{FP}, R_{FN}) = \frac{R_P - R_{FP}}{1 - R_{FN} - R_{FP}} \\
\end{align*}

Probability of Data (we have two ):
\begin{align*}
    P(D) &= P() \\
\end{align*}
for uniform $R$,
\begin{align*}
    P(k) &= \int_0^\infty \frac{1}{m-n} \frac{e^{-RT}(RT)^k}{k!} dR \\
    &= \frac{1}{m-n}\frac{1}{k!} \int_0^\infty e^{-RT}(RT)^k dR \\
    &= \frac{1}{m-n}\frac{1}{k!}\frac{1}{T} \int_0^\infty e^{-\lambda}(\lambda)^k d\lambda \\
    &= \frac{1}{m-n}\frac{1}{k!}\frac{1}{T} k! \\
    &= \frac{1}{m-n}\frac{1}{T} \\
\end{align*}
or for uniform $\log_{10}(R)$
\begin{align*}
    P(k) &= \int_0^\infty \frac{1}{\ln(m)-\ln(n)}\frac{1}{R} \frac{e^{-RT}(RT)^k}{k!} dR \\
    &= \frac{1}{\ln(m)-\ln(n)} \int_0^\infty T \frac{1}{RT} \frac{e^{-RT}(RT)^k}{k!} dR \\
    &= \frac{1}{\ln(m)-\ln(n)} \int_0^\infty \frac{e^{-\lambda}(\lambda)^{k-1}}{k!} d\lambda &[\lambda = RT]\\
    &= \frac{1}{\ln(m)-\ln(n)}\frac{1}{k} \\
\end{align*}

Now calculate posteriors:

Uniform $R$:
\begin{align*}
    P(R|k) &= \frac{P(k|R)P(R)}{P(k)}\\
    &= \frac{\frac{e^{-RT}(RT)^k}{k!}\frac{1}{m-n}}{\frac{1}{m-n}\frac{1}{T}}\\
    &= \frac{\frac{e^{-RT}(RT)^k}{k!}}{\frac{1}{T}}\\
    &= \frac{Te^{-RT}(RT)^k}{k!}\\
\end{align*}

Uniform $\log_{10}(R)$
\begin{align*}
    P(R|k) &= \frac{P(k|R)P(R)}{P(k)}\\
    &= \frac{\frac{e^{-RT}(RT)^k}{k!}\frac{1}{\ln(m)-\ln(n)} \frac{1}{R}}{\frac{1}{\ln(m)-\ln(n)}\frac{1}{k}}\\
    &= \frac{Te^{-RT}(RT)^{k-1}}{(k-1)!}\\
\end{align*}