\section{COVID-19 Study}
\begin{itemize}
    \item Sample size: 3330 people in Santa Clara County, California
    \item $N_{+,t}$ = 50 positive test results in test group
    \item Control group 1: 3324 people, definitely negative
    \item $N_{+,c1}$ = 16 positives in control 1
    \item Control group 2: 157 people, definitely negative
    \item $N_{+,c2}$ = 130 positives in control 2 (27 false negatives)
\end{itemize}

Calculate the Bayesian 95\% central interval on the fraction of people in Santa Clara County who actually had antibodies for COVID-19, marginalizing over the false positive and false negative rates. Assume flat priors on all parameters. Submit a plot of the posterior distribution for the true incidence rate as well as your code or calculation.

We want to find the distribution of the true disease in the population and then get the 95\% interval.

Relevant variables, $R_{TP}, R_{FP}, R_{TN}, R_{FN}, R_P, R_N$ ($P$=Positive, $N$=Negative, $T$=True, $F$=False).

Use Bayes to get probability distributions for 3 of these (just three since $R_P = 1 - R_N, R_{TP} = 1 - R_{FP}, R_{TN} = 1 - R_{FN}$).

Each one of these 3 follows a binomial distribution, so

\textbf{Distribution for $P(R_{FP}|\text{data from control group 1})$}:

Prior: $P(R_{FP}) = 1$

Data: in our control group 1, $N_1 = 3324, N_{FP,1} = 16$.

Likelihood (binomial): $P(N_1=3324, N_{FP,1}=16 | R_{FP}) = {N_1 \choose N_{FP,1}} R_{FP}^{N_{FP,1}} (1-R_{FP,1})^{N_1-N_{FP,1}}$

Normalization:
\begin{align*}
    P(N_1=3324, N_{FP,1}=16) = \int_0^1 dR_{FP} {N_1 \choose N_{FP,1}} R_{FP}^{N_{FP,1}} (1-R_{FP})^{N_1-N_{FP,1}} \\
\end{align*}

Posterior:
\begin{align*}
    P(R_{FP}|N_1=3324, N_{FP,1}=16) &= \frac{P(R_{FP}) P(N_1=3324, N_{FP,1}=16 | R_{FP})}{P(N_1=3324, N_{FP,1}=16)} \\
    &= \frac{R_{FP}^{N_{FP,1}} (1-R_{FP,1})^{N_1-N_{FP,1}}}{\int_0^1 dR_{FP} R_{FP}^{N_{FP,1}} (1-R_{FP})^{N_1-N_{FP,1}}} \\
\end{align*}

\textbf{Similarly, the other distributions:}
\begin{align*}
    P(R_{FN}|N_2=157, N_{FN,2}=27) 
    &= \frac{R_{FN}^{N_{FN,2}} (1-R_{FN,2})^{N_2-N_{FN,2}}}{\int_0^1 dR_{FN} R_{FN}^{N_{FN,2}} (1-R_{FN})^{N_2-N_{FN,2}}} \\
\end{align*}

\begin{align*}
    P(R_{P}| N_t=3330, N_{P,t}=50) 
    &= \frac{R_{P}^{N_{P,t}} (1-R_{P,t})^{N_t-N_{P,t}}}{\int_0^1 dR_{P} R_{P}^{N_{P,t}} (1-R_{P})^{N_t-N_{P,t}}} \\
\end{align*}

What we really want is the PDF for the probability of having antibodies $P(A)$. Let's find that by relating the things we already have, make it a variable, say $a$.

A positive test could have come from a true positive or a false positive, so
\begin{align*}
    R_{P} &= R_{TP}a + R_{FP}(1-a) \\
    \frac{R_{P} - R_{FP}}{R_{TP} - R_{FP}} &= a  \\
    \frac{R_{P} - R_{FP}}{(1-R_{FP}) - R_{FP}} &= a  \\
\end{align*}

We have PDFs for all those variables, use transformations to get the prior PDF for $a$.

\todo[inline]{How to get that? I can't use convolutions like on A1... Do I do two convolutions for numerator and denominator then something else for the ratio?}

Finally marginalize over $R_{FP}, R_{FN}$, and plot:


